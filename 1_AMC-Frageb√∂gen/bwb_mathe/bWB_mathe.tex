%---------------------------------------------
% Präambel

\documentclass[a4paper,10pt]{article}

\usepackage[utf8]{inputenc}
\usepackage[T1]{fontenc}
\usepackage[german]{babel}

\renewcommand*\rmdefault{lmss}
\usepackage{setspace}

\usepackage{color, colortbl}
\definecolor{grau}{rgb}{.9,.9,.9}

\usepackage{bbding}
\usepackage{dingbat}

\usepackage[table]{xcolor}
\usepackage{array}
\usepackage{multirow}
\usepackage{multicol}
\usepackage{graphicx}
\usepackage{fp}
\usepackage{textpos}
\usepackage{dashrule}
\usepackage{tikz}
\usepackage{paralist}


% Für Fragebogen
%---------------------------------------------
% Bienvenue Stylepaket
\usepackage[box,lang=DE,noshuffle]{automultiplechoice}
% LMES Stylepaket
\usepackage{zusatz}

%---------------------------------------------
% AMC anpassen

\AMCtext{message}{\footnotesize \color{black} Bitte Kästchen eindeutig mit dunkler Farbe (kein Bleistift) ankreuzen. Zur Korrektur falsches Kästchen vollständig ausfüllen und gewolltes ankreuzen.}
\AMCtext{draft}{}

% Vertikaler Abstand zwischen den einzelnen Fragen
\def\AMCbeginQuestion#1#2{\vspace{-0.4cm}}

%Größe Antwortboxen
%\AMCboxDimensions{size=1.7ex}

\makeatletter
\def\saveenum{\xdef\@savedenum{\the\c@enumi\relax}}
\def\resetenum{\global\c@enumi\@savedenum}
\makeatother


\setlength\parindent{0pt}


\begin{document}
\onecopy{34}{

\includegraphics[width=0.25\textwidth]{../bbst_gs_deutsch/_pics/steps} \hspace{5cm}
\includegraphics[width=0.4\textwidth]{../bbst_gs_deutsch/_pics/beberlin}

\begin{center}
	\huge \textbf{Evaluation} \\
	Berufsbegleitende Weiterbildung
\end{center}

\vspace{1cm}

Sehr geehrte Teilnehmende,
\\ \\
in der berufsbegleitenden Weiterbildung arbeiten wir regelmäßig an der Qualität unseres Angebotes und möchten Ihre Ideen und Anregungen zur Verbesserung der Qualität aufnehmen. Ihre Bewertung der Maßnahmen ist uns daher sehr wichtig.
\\ \\
Bitte füllen Sie diesen Fragebogen aus und geben ihn im Anschluss bei Ihrer Dozentin/Ihrem Dozenten ab.
\\ \\
Markieren Sie die gewünschte Antwort mit einem Kreuz im vorgegebenen Kästchen: \includegraphics[scale=0.3]{../bbst_gs_deutsch/_pics/kaestchen_kreuz} \\
Bei einem Fehler füllen Sie das falsch markierte Kästchen bitte vollständig aus: \includegraphics[scale=0.3]{../bbst_gs_deutsch/_pics/kaestchen_voll}
\\ \\
Die Auswertung erfolgt selbstverständlich anonym.
\\ \\
Vielen Dank.

\vspace{1cm}

\subsection*{Bitte wählen Sie:}

\vspace{0.5cm}

\begin{questionmult}{kurs}\scoring{v=-1}
	\begin{choices}
		\correctchoice{Ergänzungs- und Erweiterungsstudium Mathematik WB-ES Ma 17/18}\scoring{b=1}
		\correctchoice{Ergänzungs- und Erweiterungsstudium Mathematik WB-ES Ma 18/19}\scoring{b=2}
	\end{choices}
\end{questionmult}


\pagebreak

\subsection*{1. Besuchen Sie zum ersten Mal eine berufsbegleitende Weiterbildung?}
\vspace{0.5cm}
\begin{questionmult}{erstes}\scoring{v=-1}
	\begin{choiceshoriz}
		\correctchoice{Ja \small{(bitte weiter bei \textbf{Frage 3})}}\scoring{b=1}
		\correctchoice{Nein \small{(bitte weiter bei \textbf{Frage 2})}}\scoring{b=2}
	\end{choiceshoriz}
\end{questionmult}


\subsection*{2. Welche Weiterbildung haben Sie bereits absolviert?}
\vspace{0.5cm}
\begin{questionmult}{absolv}\scoring{v=-1}
	\begin{choices}
		\correctchoice{Ergänzungs- und Erweiterungsstudium Englisch}\scoring{b=1}
		\correctchoice{Ergänzungs- und Erweiterungsstudium Informatik}\scoring{b=2}
		\correctchoice{Ergänzungs- und Erweiterungsstudium Mathematik}\scoring{b=3}
		\correctchoice{Erweiterungsstudium Sonderpädagogik}\scoring{b=4}
		\correctchoice{Ergänzungs- und Erweiterungsstudium WAT}\scoring{b=5}
		\correctchoice{Weiterbildungslehrgang Englisch}\scoring{b=6}
		\correctchoice{Weiterbildungslehrgang Ethik}\scoring{b=7}
		\correctchoice{Weiterbildungslehrgang Gesellschaftswissenschaften}\scoring{b=8}
		\correctchoice{Weiterbildungslehrgang Mathematik}\scoring{b=9}
		\correctchoice{Weiterbildungslehrgang Naturwissenschaften}\scoring{b=10}
		\correctchoice{Weiterbildungslehrgang Psychologie}\scoring{b=11}
		\correctchoice{Weiterbildungslehrgang Schwimmen}\scoring{b=12}
		\correctchoice{Weiterbildungslehrgang Theater/Darstellendes Spiel}\scoring{b=13}
		\correctchoice{Qualifizierung Beratungslehrkraft für den schulpsychologischen Dienst (BSD)}\scoring{b=14}
		\correctchoice{Qualifizierung Deutsche Gebärdensprache}\scoring{b=15}
		\correctchoice{Qualifizierung Sprachbildungskoordinatorin bzw. Sprachbildungskoordinator}\scoring{b=16}
		\correctchoice{Qualifizierung Durchgängige Sprachbildung}\scoring{b=17}
		\correctchoice{Qualifizierung Unterrichts- und Schulentwicklung für die inklusive Schule}\scoring{b=18}
		\correctchoice{Sonderpädagogische Zusatzausbildung für Pädagogische Unterrichtshilfen}\scoring{b=19}
		\correctchoice{Zusatzqualifikation Facherzieherin und Facherzieher für Integration}\scoring{b=20}
		\correctchoice{andere}\scoring{b=21}
	\end{choices}
\end{questionmult}


\subsection*{3. Wie wurden Sie auf diese Weiterbildungsmöglichkeit aufmerksam? \\ (Mehrfachantworten möglich)}
\vspace{0.5cm}
\begin{questionmult}{aufmerk_weit}\scoring{v=-1}
	\begin{choices}
		\correctchoice{durch den Besuch einer anderen Weiterbildung/Fortbildung}\scoring{b=1}
		\correctchoice{Eigenrecherche/ Interesse}\scoring{b=2}
		\correctchoice{Information durch Schulleitung}\scoring{b=3}
		\correctchoice{Internet/Intranet}\scoring{b=4}
		\correctchoice{Kollegium}\scoring{b=5}
		\correctchoice{Rundschreiben}\scoring{b=6}
		\correctchoice{sonstiges}\scoring{b=7}
	\end{choices}
\end{questionmult}



\subsection*{4. Aus welchen Gründen haben Sie sich für diese Weiterbildung angemeldet? \\ (Mehrfachantworten möglich)}
\vspace{0.5cm}
\begin{questionmult}{gruend}\scoring{v=-1}
	\begin{choices}
		\correctchoice{weil ich mich persönlich weiterentwickeln möchte}\scoring{b=1}
		\correctchoice{weil ich mich beruflich weiterbilden möchte}\scoring{b=2}
		\correctchoice{weil ich dieses Fach bereits fachfremd unterrichte}\scoring{b=3}
		\correctchoice{zur Höhergruppierung}\scoring{b=4}
		\correctchoice{weil diese Qualifikation an unserer Schule gebraucht wird}\scoring{b=5}
		\correctchoice{weil diese Qualifikation zu unserem Schulprofil passt}\scoring{b=6}
		\correctchoice{andere Gründe}\scoring{b=7}
	\end{choices}
\end{questionmult}


\subsection*{5. Sind Sie insgesamt mit der Weiterbildung zufrieden?}
\hspace{2.4cm} sehr \hspace{7.8cm} gar nicht \\ \\
\begin{questionmult}{zufried}\scoring{v=-1}
	\begin{choiceshoriz}
		\correctchoice{}\scoring{b=1}
		\correctchoice{}\scoring{b=2}
		\correctchoice{}\scoring{b=3}
		\correctchoice{}\scoring{b=4}
		\correctchoice{}\scoring{b=5}
		\correctchoice{}\scoring{b=6}
	\end{choiceshoriz}
\end{questionmult}


\subsection*{6.Wie viele Stunden wenden Sie für diese Weiterbildung pro Woche auf \\ (Vor- und Nachbereitung etc., ohne Präsenzzeiten in der Veranstaltung)?}
\vspace{.5cm}
\begin{questionmult}{v6}\scoring{v=-1}
	\begin{choiceshoriz}
		\correctchoice{1}\scoring{b=1}
		\correctchoice{2}\scoring{b=2}
		\correctchoice{3}\scoring{b=3}
		\correctchoice{4}\scoring{b=4}
		\correctchoice{5}\scoring{b=5}
		\correctchoice{6}\scoring{b=6}
		\correctchoice{mehr}\scoring{b=7}
	\end{choiceshoriz}
\end{questionmult}


\pagebreak


\subsection*{7. Fragen zu Struktur und Ablauf der Fachvorlesung}
\tikzmark{f4a} % Auskommentieren, falls keine Trennlinie

\headerfive{6.7cm}{\small trifft voll zu}{\small trifft überwiegend zu}{\small trifft weniger zu}{\small trifft gar nicht zu}{keine Angabe} \\

\begin{enumerate}

\itemsfive{fv_1}{6cm}{\item Die Planung der Lehrveranstaltungen ist klar und übersichtlich.}{1}{2}{3}{4}{-1} \\
\itemsfive{fv_2}{6cm}{\item Der Ablauf der Lehrveranstaltungen entspricht der Ankündigung.}{1}{2}{3}{4}{-1} \\
\itemsfive{fv_3}{6cm}{\item Die Gestaltung der Lehrveranstaltungen ist zielführend.}{1}{2}{3}{4}{-1} \\
\itemsfive{fv_4}{6cm}{\item Die Veranstaltungszeit der Lehrveranstaltungen wird effizient genutzt.}{1}{2}{3}{4}{-1} \\
\itemsfive{fv_5}{6cm}{\item Die Gruppengröße ist für die Fachvorlesung angemessen.}{1}{2}{3}{4}{-1} \\
\itemsfive{fv_6}{6cm}{\item Die Lehrveranstaltungen beginnen und enden pünktlich.}{1}{2}{3}{4}{-1} \\
\itemsfive{fv_7}{6cm}{\item Die Teilnehmenden erscheinen pünktlich.}{1}{2}{3}{4}{-1} \\
\textbf{Materialien} \\ \\
\itemsfive{fv_8}{6cm}{\item Das Ausbildungsmaterial ist aktuell und informativ.}{1}{2}{3}{4}{-1} \\
\itemsfive{fv_9}{6cm}{\item Über Literatur und zusätzliche Materialien wird informiert.}{1}{2}{3}{4}{-1}

\end{enumerate} \saveenum

\vspace{-1cm}
\tikzmark{f4b}

\gestrichelt{13.7}{13.7}{f4a}{f4b} % manuell positionieren

\pagebreak


\subsection*{8. Fragen zu Struktur und Ablauf der Fachdidaktik-Seminare}

\tikzmark{f4a} % Auskommentieren, falls keine Trennlinie

\headerfive{6.7cm}{\small trifft voll zu}{\small trifft überwiegend zu}{\small trifft weniger zu}{\small trifft gar nicht zu}{keine Angabe} \\

\begin{enumerate}

\itemsfive{fd_1}{6cm}{\item Die Planung der Lehrveranstaltungen ist klar und übersichtlich.}{1}{2}{3}{4}{-1} \\
\itemsfive{fd_2}{6cm}{\item Der Ablauf der Lehrveranstaltungen entspricht der Ankündigung.}{1}{2}{3}{4}{-1} \\
\itemsfive{fd_3}{6cm}{\item Die Gestaltung der Lehrveranstaltungen ist abwechslungsreich.}{1}{2}{3}{4}{-1} \\
\itemsfive{fd_4}{6cm}{\item Die/der Dozierende sorgt dafür, dass alle Teilnehmenden aktiv teilnehmen können.}{1}{2}{3}{4}{-1} \\
\itemsfive{fd_5}{6cm}{\item Die/der Dozierende sorgt für Transparenz in Bezug auf Leistungsanforderungen und -bewertung.}{1}{2}{3}{4}{-1} \\
\itemsfive{fd_6}{6cm}{\item Fragen, Erfahrungen und Anregungen der Teilnehmenden werden in den Veranstaltungen aufgegriffen.}{1}{2}{3}{4}{-1} \\
\itemsfive{fd_7}{6cm}{\item Es gibt ausreichend Übungsmöglichkei\-ten.}{1}{2}{3}{4}{-1} \\
\itemsfive{fd_8}{6cm}{\item Die Lehrveranstaltungen beginnen und enden pünktlich.}{1}{2}{3}{4}{-1} \\
\itemsfive{fd_9}{6cm}{\item Die Teilnehmenden erscheinen pünktlich.}{1}{2}{3}{4}{-1} \\
\smallskip
\textbf{Materialien} \\ \\
\itemsfive{fd_10}{6cm}{\item Das Ausbildungsmaterial ist aktuell und informativ.}{1}{2}{3}{4}{-1} \\
\itemsfive{fd_11}{6cm}{\item Über Literatur und zusätzliche Materialien wird informiert.}{1}{2}{3}{4}{-1} \\
\smallskip
\textbf{Kompetenzerweiterung} \\ \\
\itemsfive{fd_12}{6cm}{\item Das vermittelte und erworbene Fachwissen ist als Grundlagenwissen relevant.}{1}{2}{3}{4}{-1} \\
\itemsfive{fd_13}{6cm}{\item Das Fachwissen stellt eine Unterstützung bei der Unterrichtsvorbereitung dar.}{1}{2}{3}{4}{-1}

\end{enumerate} \saveenum

\vspace{-1cm}
\tikzmark{f4b}

\gestrichelt{13.7}{13.7}{f4a}{f4b} % manuell positionieren


\pagebreak


\subsection*{9. Fragen zu Struktur und Ablauf der (Präsenz-) Übungen}

\tikzmark{f4a} % Auskommentieren, falls keine Trennlinie

\headerfive{6.7cm}{\small trifft voll zu}{\small trifft überwiegend zu}{\small trifft weniger zu}{\small trifft gar nicht zu}{keine Angabe} \\

\begin{enumerate}

\itemsfive{ue_1}{6cm}{\item Die Planung der Lehrveranstaltungen ist klar und übersichtlich.}{1}{2}{3}{4}{-1} \\
\itemsfive{ue_2}{6cm}{\item Der Ablauf der Lehrveranstaltungen entspricht der Ankündigung.}{1}{2}{3}{4}{-1} \\
\itemsfive{ue_3}{6cm}{\item Die Gestaltung der Lehrveranstaltungen ist zielführend.}{1}{2}{3}{4}{-1} \\
\itemsfive{ue_4}{6cm}{\item Die Inhalte der Übungen sind sinnvoll auf die Fachvorlesung abgestimmt.}{1}{2}{3}{4}{-1} \\
\itemsfive{ue_5}{6cm}{\item Fragen, Erfahrungen und Anregungen der Teilnehmenden werden in den Veranstaltungen aufgegriffen.}{1}{2}{3}{4}{-1} \\
\itemsfive{ue_6}{6cm}{\item Es gibt genügend Zeit für den allgemeinen/ fachlichen Austausch.}{1}{2}{3}{4}{-1} \\
\itemsfive{ue_7}{6cm}{\item Die Gruppengröße ist für die Lehrveranstaltungen angemessen.}{1}{2}{3}{4}{-1} \\
\itemsfive{ue_8}{6cm}{\item Die Lehrveranstaltungen beginnen und enden pünktlich.}{1}{2}{3}{4}{-1} \\
\itemsfive{ue_9}{6cm}{\item Die Teilnehmenden erscheinen pünktlich.}{1}{2}{3}{4}{-1} \\
\textbf{Materialien} \\ \\
\itemsfive{ue_10}{6cm}{\item Das Ausbildungsmaterial ist aktuell und informativ.}{1}{2}{3}{4}{-1} \\
\itemsfive{ue_11}{6cm}{\item Über Literatur und zusätzliche Materialien wird informiert.}{1}{2}{3}{4}{-1}

\end{enumerate} \saveenum

\vspace{-1cm}
\tikzmark{f4b}

\gestrichelt{13.7}{13.7}{f4a}{f4b} % manuell positionieren


\pagebreak


\subsection*{10. Fragen zu Struktur und Ablauf des Seminars}

\tikzmark{f4a} % Auskommentieren, falls keine Trennlinie

\headerfive{6.7cm}{\small trifft voll zu}{\small trifft überwiegend zu}{\small trifft weniger zu}{\small trifft gar nicht zu}{keine Angabe} \\

\begin{enumerate}

\itemsfive{se_ma_1}{6cm}{\item Die Planung der Lehrveranstaltungen ist klar und übersichtlich.}{1}{2}{3}{4}{-1} \\
\itemsfive{se_ma_2}{6cm}{\item Der Ablauf der Lehrveranstaltungen entspricht der Ankündigung.}{1}{2}{3}{4}{-1} \\
\itemsfive{se_ma_3}{6cm}{\item Die Gestaltung der Lehrveranstaltungen ist zielführend.}{1}{2}{3}{4}{-1} \\
\itemsfive{se_ma_4}{6cm}{\item Die Inhalte des Seminars führen zu einem vertieften Einblick in bestimmte weiterführende Themen der Mathematik.}{1}{2}{3}{4}{-1} \\
\itemsfive{se_ma_5}{6cm}{\item Fragen, Erfahrungen und Anregungen der Teilnehmenden werden in den Veranstaltungen aufgegriffen.}{1}{2}{3}{4}{-1} \\
\itemsfive{se_ma_6}{6cm}{\item Es gibt genügend Zeit für den allgemeinen/ fachlichen Austausch.}{1}{2}{3}{4}{-1} \\
\itemsfive{se_ma_7}{6cm}{\item Die Gruppengröße ist für die Lehrveranstaltungen angemessen.}{1}{2}{3}{4}{-1} \\
\itemsfive{se_ma_8}{6cm}{\item Die Lehrveranstaltungen beginnen und enden pünktlich.}{1}{2}{3}{4}{-1} \\
\itemsfive{se_ma_9}{6cm}{\item Die Teilnehmenden erscheinen pünktlich.}{1}{2}{3}{4}{-1} \\
\textbf{Materialien} \\ \\
\itemsfive{se_ma_10}{6cm}{\item Das Ausbildungsmaterial ist aktuell und informativ.}{1}{2}{3}{4}{-1} \\
\itemsfive{se_ma_11}{6cm}{\item Über Literatur und zusätzliche Materialien wird informiert.}{1}{2}{3}{4}{-1}

\end{enumerate} \saveenum

\vspace{-1cm}
\tikzmark{f4b}

\gestrichelt{13.7}{13.7}{f4a}{f4b} % manuell positionieren

\pagebreak



\subsection*{11. Würden Sie diese Weiterbildung weiterempfehlen?}
\vspace{0.5cm}
\begin{questionmult}{empfehl}\scoring{v=-1}
	\begin{choiceshoriz}
		\correctchoice{Ja}\scoring{b=1}
		\correctchoice{Nein}\scoring{b=2}
	\end{choiceshoriz}
\end{questionmult}


\subsection*{12. Interessieren Sie sich außerdem für eine andere Weiterbildung?}
\vspace{0.5cm}
\begin{questionmult}{andere}\scoring{v=-1}
	\begin{choices}
		\correctchoice{Kein Interesse}\scoring{b=0} \\
		\correctchoice{Deutsche Gebärdensprache}\scoring{b=1}
		\correctchoice{Englisch}\scoring{b=2}
		\correctchoice{Ethik}\scoring{b=3}
		\correctchoice{Facherzieherin und Facherzieher für Integration}\scoring{b=4}
		\correctchoice{Geschichte}\scoring{b=5}
		\correctchoice{Gesellschaftswissenschaften}\scoring{b=6}
		\correctchoice{Informatik}\scoring{b=7}
		\correctchoice{Inklusion}\scoring{b=8}
		\correctchoice{Mathematik}\scoring{b=9}
		\correctchoice{Musik}\scoring{b=10}
		\correctchoice{Naturwissenschaften}\scoring{b=11}
		\correctchoice{Pädagogische Unterrichtshilfen}\scoring{b=12}
		\correctchoice{Philosophie}\scoring{b=13}
		\correctchoice{Politik}\scoring{b=14}
		\correctchoice{Psychologie/Schulpsychologie}\scoring{b=15}
		\correctchoice{Schwimmen}\scoring{b=16}
		\correctchoice{Sprachbildung}\scoring{b=17}
		\correctchoice{Sonderpädagogik}\scoring{b=18}
		\correctchoice{Theater/Darstellendes Spiel}\scoring{b=19}
		\correctchoice{WAT}\scoring{b=20}
		\correctchoice{andere}\scoring{b=21}
	\end{choices}
\end{questionmult}

\pagebreak

\section*{Für die Auswertung benötigen wir noch einige allgemeine \mbox{Angaben}}

\subsection*{13. Geschlecht}
\vspace{0.5cm}
\begin{questionmult}{geschl}\scoring{v=-1}
	\begin{choiceshoriz}
		\correctchoice{männlich}\scoring{b=1}
		\correctchoice{weiblich}\scoring{b=2}
	\end{choiceshoriz}
\end{questionmult}

\subsection*{14. Alter}
\vspace{0.5cm}
\begin{questionmult}{alter}\scoring{v=-1}
	\begin{choices}
		\correctchoice{30 Jahre und jünger}\scoring{b=1}
		\correctchoice{31 bis 40 Jahre}\scoring{b=2}
		\correctchoice{41 bis 50 Jahre}\scoring{b=3}
		\correctchoice{51 bis 60 Jahre}\scoring{b=4}
		\correctchoice{61 Jahre und älter}\scoring{b=5}
	\end{choices}
\end{questionmult}

\subsection*{15. Schultyp}
\vspace{0.5cm}
\begin{questionmult}{schulytyp}\scoring{v=-1}
	\begin{choices}
		\correctchoice{Grundschule}\scoring{b=1}
		\correctchoice{Integrierte Sekundarschule}\scoring{b=2}
		\correctchoice{Gymnasium}\scoring{b=3}
		\correctchoice{berufliche Schule}\scoring{b=4}
		\correctchoice{Schule mit sonderpädagogischem Förderschwerpunkt}\scoring{b=5}
		\correctchoice{Gemeinschaftsschule}\scoring{b=6}
	\end{choices}
\end{questionmult}

\subsection*{16. Einsatz meist in Klassenstufe: \\ (Mehrfachantworten möglich)}
\vspace{0.5cm}
\begin{questionmult}{klassen}\scoring{v=-1}
	\begin{choices}
		\correctchoice{Klassenstufe 1 bis 3}\scoring{b=1}
		\correctchoice{Klassenstufe 4 bis 6 }\scoring{b=2}
		\correctchoice{Klassenstufe 7 bis 10}\scoring{b=3}
		\correctchoice{Klassenstufe 11 bis 12/13}\scoring{b=4}
	\end{choices}
\end{questionmult}



\bigskip

\begin{flushleft}
\textbf{Vielen Dank, dass Sie sich die Zeit genommen haben. Sie haben uns mit Ihren Auskünften sehr geholfen. Viel Erfolg und Freude bei Ihrer weiteren beruflichen Tätigkeit!}
\end{flushleft}

\newpage
\thispagestyle{empty}
\quad  \addtocounter{page}{-1}
\newpage


}
\end{document}
