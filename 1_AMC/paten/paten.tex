
% Präambel

\documentclass[a4paper]{article}
\usepackage[utf8]{inputenc}
\usepackage[T1]{fontenc}
\usepackage[german]{babel}

\renewcommand*\rmdefault{lmss}
\usepackage{setspace}

\usepackage{color, colortbl}
\definecolor{grau}{rgb}{.9,.9,.9}

\usepackage{bbding}
\usepackage{dingbat}

\usepackage[table]{xcolor}
\usepackage{array}
\usepackage{multirow}
\usepackage{multicol}
\usepackage{graphicx}
\usepackage{fp}
\usepackage{textpos}
\usepackage{dashrule}
\usepackage{tikz}
\usepackage{paralist}

% Für Fragebogen
%---------------------------------------------
% Bienvenue Stylepaket
\usepackage[box,lang=DE,noshuffle]{automultiplechoice}
% LMES Stylepaket
\usepackage{zusatz}

%---------------------------------------------
% AMC anpassen

\AMCtext{message}{\footnotesize \color{gray} Bitte Kästchen eindeutig mit dunkler Farbe (kein Bleistift) ankreuzen. Zur Korrektur falsches Kästchen vollständig ausfüllen und gewolltes ankreuzen.}
\AMCtext{draft}{}

% Vertikaler Abstand zwischen den einzelnen Fragen
\def\AMCbeginQuestion#1#2{\vspace{-0.4cm}}

%Größe Antwortboxen
%\AMCboxDimensions{size=1.7ex}

\makeatletter
\def\saveenum{\xdef\@savedenum{\the\c@enumi\relax}}
\def\resetenum{\global\c@enumi\@savedenum}
\makeatother


\setlength\parindent{0pt}

\begin{document}

\onecopy{30}{

\vspace{-0.5cm}

\includegraphics[width=0.25\textwidth]{../bbst_gs_deutsch/_pics/arrow} \hspace{4.8cm}
\includegraphics[width=0.4\textwidth]{../bbst_gs_deutsch/_pics/beberlin}

\vspace{.5cm}

\begin{center}
	\textbf{\huge \scshape Evaluation} \\ \vspace{0.3cm}
	\textbf{\Large \scshape FIRST STEPS-Paten im QuerBer}
\end{center}

\vspace{1.5cm}

Mein Pate/meine Patin: \_\_\_\_\_\_\_\_\_\_\_\_\_\_\_\_\_\_\_\_\_ (freiwillige Angabe)

\vspace{1.5cm}

\begin{tabular}{ll}

Ich habe bereits vor dem KICK OFF an einer Schule gearbeitet. &

\begin{questionmult}{v0}\scoring{v=-1}
	\begin{choices}
		\correctchoice{Ja}\scoring{b=1}
		\correctchoice{Nein}\scoring{b=2}
	\end{choices}
\end{questionmult}

\end{tabular}

\vspace{1.5cm}

\subsection*{Fragen zum Paten/zur Patin}

\vspace{-1.3cm}

\begin{tabular}{p{7cm} l p{1cm} r}
	& Trifft voll und & & Trifft überhaupt \\
	& ganz zu 				& & nicht zu
\end{tabular}

\vspace{-.1cm}

\tikzmark{a}

\headerseven{}{1}{2}{3}{4}{5}{6}{weiß nicht} \\

\begin{enumerate}

	\itemsseven{v1}{\item Mein Pate/meine Patin ist auf meine \newline Fragen und Wünsche eingegangen.} \\
	\itemsseven{v2}{\item Die Gespräche mit meinem Paten/meiner Patin fanden auf Augenhöhe statt.} \\
	\itemsseven{v3}{\item Die Gespräche mit meinem Paten/meiner Patin fanden in einer ruhigen und vertraulichen Atmosphäre statt.} \\
	\itemsseven{v4}{\item Mein Pate/meine Patin gab mir hilfreiches und konstruktives Feedback.} \\
	\itemsseven{v5}{\item Die Gespräche mit meinem Paten/meiner Patin haben mich gestärkt.} \\
	\itemsseven{v6}{\item Mein Pate/meine Patin hat NICHT aktiv in den Unterricht eingegriffen.} \\
	\itemsseven{v7}{\item Die Kommunikation mit meinem \newline Paten/meiner Patin hat gut funktioniert.} \\
	\itemsseven{v8}{\item Mein Pate/meine Patin konnte mich bei der Rollenfindung als Lehrkraft unterstützen.} \\
	\itemsseven{v9}{\item Dank der Unterstützung meines \newline Paten/meiner Patin fühlte ich mich meiner Aufgabe besser gewachsen.} \\
	\itemsseven{v10}{\item Insgesamt würde ich meinen Paten/meine Patin weiterempfehlen.}

\end{enumerate}\saveenum

\vspace{-1cm}
\tikzmark{b}

\gestrichelt{14.15}{14.15}{a}{b} % manuell positionieren


\pagebreak


\subsection*{Fragen zum Patensystem}

\vspace{-1.3cm}

\begin{tabular}{p{7cm} l p{1cm} r}
	& Trifft voll und & & Trifft überhaupt \\
	& ganz zu 				& & nicht zu
\end{tabular}

\vspace{-.1cm}

\tikzmark{a}

\headerseven{}{1}{2}{3}{4}{5}{6}{weiß nicht} \\

\begin{enumerate}\resetenum

	\itemsseven{v11}{\item Ich konnte die erlernten Inhalte des KICK OFF anwenden.} \\
	\itemsseven{v12}{\item Die Schulleitung war ausreichend über das Patensystem informiert.} \\
	\itemsseven{v13}{\item Die Schulleitung hat sich Zeit für ein gemeinsames Gespräch mit mir und meinem Paten/meiner Patin genommen.} \\
	\itemsseven{v14}{\item Die Ziele der Patenschaft und wie sie organisiert ist, wurden verständlich vermittelt.} \\
	\itemsseven{v15}{\item Mir war bewusst, welche Rolle mein \newline Pate/meine Patin bei meinem Einstieg \newline hatte.} \\
	\itemsseven{v16}{\item Es gab ausreichend Zeit, mich mit meinem Paten/meiner Patin auszutauschen.} \\
	\itemsseven{v17}{\item Die Dauer der Patenschaft (8 Wochen) war angemessen.}

\end{enumerate}\saveenum

\vspace{-1cm}
\tikzmark{b}

\gestrichelt{14.15}{14.15}{a}{b}


\begin{tabular}{p{1cm}p{1.5cm}l}

  & Falls nein: &

  \begin{questionmult}{v17_no}\scoring{v=-1}
    \begin{choices}
    	\correctchoice{länger}\scoring{b=1}
    	\correctchoice{kürzer}\scoring{b=2}
    \end{choices}
  \end{questionmult}

\end{tabular}


\vspace{0.3cm}

\tikzmark{a}

\begin{enumerate}\resetenum

	\itemsseven{v18}{\item Die Patenschaft wurde im richtigen Zeitfenster durchgeführt.}

\end{enumerate}\saveenum

\vspace{-1.2cm}
\tikzmark{b}

\gestrichelt{14.15}{14.15}{a}{b}

\vspace{0.3cm}

\begin{tabular}{p{1cm}p{1.5cm}l}

  & Falls nein: &

\begin{questionmult}{v18_no}\scoring{v=-1}
	\begin{choices}
		\correctchoice{besser früher}\scoring{b=1}
		\correctchoice{besser später}\scoring{b=2}
	\end{choices}
\end{questionmult}

\end{tabular}


\vspace{0.4cm}

\tikzmark{a}

\begin{enumerate}\resetenum

	\itemsseven{v19}{\item Insgesamt würde ich zukünftigen Quereinsteigern das Konzept der Patenschaft \newline weiterempfehlen.}

\end{enumerate}

\vspace{-1.2cm}
\tikzmark{b}

\gestrichelt{14.15}{14.15}{a}{b} % manuell positionieren


\vspace{1cm}

\subsection*{Haben Sie noch Anregungen, Ideen, Verbesserungsvorschläge, Wünsche?}

\begin{tabular}{| p{14.5cm} |}
  	\cline{1-1} \\ \\ \\ \\ \\ \\ \\ \\ \\ \\ \\ \cline{1-1}
\end{tabular}

}

\end{document}
