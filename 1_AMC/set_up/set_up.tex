%---------------------------------------------
% Präambel

\documentclass[a4paper]{article}
\usepackage[utf8]{inputenc}
\usepackage[T1]{fontenc}
\usepackage[german]{babel}

\renewcommand*\rmdefault{lmss}
\usepackage{setspace}

\usepackage{color, colortbl}
\definecolor{grau}{rgb}{.9,.9,.9}

\usepackage{bbding}
\usepackage{dingbat}

\usepackage[table]{xcolor}
\usepackage{array}
\usepackage{multirow}
\usepackage{multicol}
\usepackage{graphicx}
\usepackage{fp}
\usepackage{textpos}
\usepackage{dashrule}
\usepackage{tikz}
\usepackage{paralist}

% Für Fragebogen
%---------------------------------------------
% Bienvenue Stylepaket
\usepackage[box,lang=DE,noshuffle]{automultiplechoice}
% LMES Stylepaket
\usepackage{zusatz}

%---------------------------------------------
% AMC anpassen

\AMCtext{message}{\footnotesize \color{gray} Bitte Kästchen eindeutig mit dunkler Farbe (kein Bleistift) ankreuzen. Zur Korrektur falsches Kästchen vollständig ausfüllen und gewolltes ankreuzen.}
\AMCtext{draft}{}

% Vertikaler Abstand zwischen den einzelnen Fragen
\def\AMCbeginQuestion#1#2{\vspace{-0.4cm}}

%Größe Antwortboxen
%\AMCboxDimensions{size=1.7ex}

\makeatletter
\def\saveenum{\xdef\@savedenum{\the\c@enumi\relax}}
\def\resetenum{\global\c@enumi\@savedenum}
\makeatother


\setlength\parindent{0pt}

\onecopy{1}{


\begin{document}

\vspace{-0.5cm}

\includegraphics[width=0.25\textwidth]{../bbst_gs_deutsch/_pics/steps} \hspace{4.9cm}
\includegraphics[width=0.4\textwidth]{../bbst_gs_deutsch/_pics/beberlin}

\vspace{.5cm}

\begin{center}
	\textbf{\Huge \scshape Feedback} \\ \vspace{0.3cm}
	\textbf{\huge \scshape zum SET UP}
\end{center}

\vspace{.9cm}

\begin{tabular}{rlrl}
Seminar (Thema): & \_\_\_\_\_\_\_\_\_\_\_\_\_\_\_\_\_\_ &
SU-ID: 					 & \_\_\_\_\_\_\_\_\_\_\_\_\_\_\_\_\_\_ \\
 & \\
 & \\
Veranstaltungszeit: & \_\_\_\_\_\_\_\_\_\_\_\_\_\_\_\_\_\_ &
Name DozentIn: 			& \_\_\_\_\_\_\_\_\_\_\_\_\_\_\_\_\_\_
\end{tabular}

\vspace{.5cm}

\subsection*{Zusammenfassende Einschätzung \\ zur Veranstaltung}

\vspace{-1.3cm}

\begin{tabular}{p{7cm} l p{1cm} r}
	& Trifft voll und & & Trifft überhaupt \\
	& ganz zu 				& & nicht zu
\end{tabular}

\vspace{-.1cm}

\tikzmark{a}

\headerseven{}{1}{2}{3}{4}{5}{6}{weiß nicht} \\

\begin{enumerate}

	\itemsseven{v1}{\item Die Veranstaltung war klar und übersichtlich strukturiert.} \\
	\itemsseven{v2}{\item Die Veranstaltung versetzt mich in die Lage, die Inhalte selbständig zu vertiefen.} \\
	\itemsseven{v3}{\item Das fachliche Niveau der Veranstaltung empfand ich als angemessen.} \\
	\itemsseven{v4}{\item Das Geschehen in der Veranstaltung entsprach der Ankündigung auf dem \linebreak Bildungsserver.} \\
	\itemsseven{v5}{\item Das Erlernte und Erfahrene sind für mich beruflich von Nutzen.} \\
	\itemsseven{v6}{\item Es gab genügend Zeit für den allgemeinen bzw. fachlichen Austausch.} \\
	\itemsseven{v7}{\item Die Lehrveranstaltung begann und endete pünktlich.} \\
	\itemsseven{v8}{\item Die Veranstaltung würde ich weiter\-empfehlen.}

\end{enumerate}\saveenum

\vspace{-1cm}
\tikzmark{b}

\gestrichelt{14.15}{14.15}{a}{b} % manuell positionieren

\vspace{-.3cm}

\subsection*{Der Dozent/ die Dozentin\dots}

\vspace{-1cm}

\tikzmark{a}

\headerseven{}{1}{2}{3}{4}{5}{6}{weiß nicht} \\

\begin{enumerate}\resetenum

	\itemsseven{v9}{\item \dots hat Ziele und Struktur der Veranstaltung nachvollziehbar dargestellt.} \\
	\itemsseven{v10}{\item \dots ging, soweit wie möglich, auf Wünsche und Fragen der Teilnehmenden ein und hat Anregungen aufgegriffen.} \\
	\itemsseven{v11}{\item \dots gestaltete die Lehrveranstaltung \linebreak interessant und abwechslungsreich.}

\end{enumerate}\saveenum

\vspace{-1cm}
\tikzmark{b}

\gestrichelt{14.15}{14.15}{a}{b} % manuell positionieren


\vspace{-1cm}

\pagebreak

\subsection*{Anmerkungen}

\begin{enumerate}\resetenum

\item Weitere Hinweise, Wünsche und Anregungen:	 \\

\begin{tabular}{| p{13cm} |}
  	\cline{1-1} \\ \\ \\ \\ \\ \\ \\ \\ \\ \\ \\ \cline{1-1}
\end{tabular}

\end{enumerate}\saveenum

\end{document}

}
